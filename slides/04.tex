% -*- TeX-engine: luatex -*-
\documentclass[dvipsnames,presentation,aspectratio=169,14pt]{beamer}
\usepackage{hastingstheme}
\titlegraphic{\includegraphics[scale=.35]{static_figures/du_bn.pdf}}
\author{\large Massimiliano Fasi}
\date{}


% \usepackage{template}
% \input{setup-lawrence}
% \renewcommand{\sessionnumber}{4}
% \renewcommand{\sessiontitle}{Performance measurements}
% \usepackage{tikz}

\usepackage{pgfplots}
\pgfplotsset{compat=1.15}
\usetikzlibrary{matrix,fit,positioning,calc}
\usepackage{pgfplotstable}
\usetikzlibrary{pgfplots.groupplots}
\date{}

\begin{document}

\title{\firasemibold\color{White}%
  {\fontsize{20}{0}\selectfont SESSION 4\\
    \fontsize{40}{40}\selectfont Performance\\measurements\par}}
\titleslide

% \begin{frame}
%   \frametitle{Roofline of dense matrix--vector product}
%   \begin{center}
%     \includegraphics[width=0.9\textwidth]{figures/roofline-dmvm}
%   \end{center}
% \end{frame}

\begin{frame}
  \frametitle{Overview}
  \structure{Roofline}
  \begin{itemize}[itemsep=5pt]
  \item comparison between hardware configurations
  \item high-level overview of code performance
  \item general guidance for optimisation
  \end{itemize}
  \pause
  \vskip 11pt

  \structure{Measurements}
  \begin{itemize}[itemsep=5pt]
  \item to get more information about bottleneck
  \item to confirm the hypothesis formed through roofline analysis
  \end{itemize}
\end{frame}

\begin{frame}
  \frametitle{Performance measurements}
  Special purpose \structure{registers}:
  \begin{itemize}[itemsep=5pt]
  \item common in modern hardware
  \item record low-level performance events
    \begin{itemize}[itemsep=3pt]
    \item number of Flops of different type (scalar, sse, avx)
    \item cache miss/hit counts at various levels
    \item branch prediction success rate
    \item \ldots
    \end{itemize}
  \item<2> can be overwhelming
  \item<2> best used to confirm hypothesis from some model
  \end{itemize}
\end{frame}

\begin{frame}
  \frametitle{Caveats}

  \begin{itemize}[itemsep=8pt]
  \item Information about
    \begin{itemize}[itemsep=3pt]
    \item the algorithm you implemented
    \item the way you implemented it
    \item the data moved in the measured run
    \end{itemize}
  \item Does not consider
    \begin{itemize}[itemsep=3pt]
    \item potentially better algorithms
    \item potentially superior ways of implementing those
    \item data you could have moved in a different run
    \end{itemize}
  \end{itemize}

  \pause
  \vskip 5pt

  Only meaningful as complements to models.
\end{frame}

\begin{frame}
  \frametitle{Granularity}
  \begin{itemize}[itemsep=8pt]
  \item Direct read of low-level hardware counters
    \begin{itemize}[itemsep=4pt]
    \item most detailed
    \item hardware dependent
    \item not portable
    \end{itemize}
  \item Abstract metrics
    \begin{itemize}[itemsep=4pt]
    \item groups of low-level counters
    \item easier to compare across hardware
    \end{itemize}
  \end{itemize}

  \vskip 11pt

  ``instructions'' $\rightarrow$ ``instructions per cycle''

\end{frame}

\begin{frame}
  \frametitle{How do we measure them?}
  \begin{itemize}
  \item Use \texttt{likwid-perfctr} (installed on Hamilton via the
    \texttt{likwid} module).
  \item Offers a reasonably friendly command-line interface.
  \item Provides access both to counters directly, and many useful
    predefined ``groups''.
  \end{itemize}
  \begin{itemize}
  \item Will use \texttt{likwid-perfctr} to measure memory references
    in different implementations of the same loop.
  \end{itemize}

\end{frame}

\begin{frame}[fragile]
  \frametitle{Example: STREAM}
  \vskip -20pt
  \begin{columns}[t]
    \begin{column}{0.4\textwidth}
      \begin{block}{\footnotesize Scalar}
        \vspace{-5pt}
\begin{minted}[fontsize=\footnotesize]{asm}
for i from 0 to n:
  load a[i:i+1] reg1
  load b[i:i+1] reg2
  load c[i:i+1] reg4
  mul reg1 reg2 reg3
  add reg4 reg3 reg4
  store reg4 c[i:i+1]
\end{minted}
        \vspace{-5pt}
      \end{block}
    \end{column}
    \begin{column}{0.4\textwidth}
      \begin{block}{\footnotesize SSE}
        \vspace{-5pt}
\begin{minted}[fontsize=\footnotesize]{asm}
for i from 0 to n by 2:
  vload a[i:i+2] vreg1
  vload b[i:i+2] vreg2
  vload c[i:i+2] vreg4
  vmul vreg1 vreg2 vreg3
  vadd reg4 reg3 reg4
  vstore reg4 c[i:i+2]
\end{minted}
        \vspace{-5pt}
      \end{block}
    \end{column}
  \end{columns}
  \vspace{-5pt}
\begin{columns}[t]
  \begin{column}{0.4\textwidth}
    \begin{block}{\footnotesize AVX}
      \vspace{-5pt}
\begin{minted}[fontsize=\footnotesize]{asm}
for i from 0 to n by 4:
  vload a[i:i+4] vreg1
  vload b[i:i+4] vreg2
  vload c[i:i+4] vreg4
  vmul vreg1 vreg2 vreg3
  vadd reg4 reg3 reg4
  vstore reg4 c[i:i+4]
\end{minted}
      \vspace{-5pt}
      \end{block}
    \end{column}
    \begin{column}{0.4\textwidth}
      \begin{block}{\footnotesize AVX2}
        \vspace{-5pt}
\begin{minted}[fontsize=\footnotesize]{asm}
for i from 0 to n by 4:
  vload a[i:i+4] vreg1
  vload b[i:i+4] vreg2
  vload c[i:i+4] vreg3
  vfma vreg1 vreg2 vreg3
  vstore reg3 c[i:i+4]
\end{minted}
        \vspace{-5pt}
      \end{block}
    \end{column}
  \end{columns}
\end{frame}

\begin{frame}
  \frametitle{Measurement}
  \begin{challenge}{Model}
    For $N = \mathsf{10^6}$, how many loads and stores in each case?
  \end{challenge}
  \pause
  \begin{answer}{Answer}
    Each loop iteration has 3 loads and 1 store.

    With vector width $W$ and $N$ iterations we need:
    \vskip 4pt
    \begin{itemize}[itemsep=4pt]
    \item $\dfrac{\mathsf 3N}{W}$ loads
    \item $\dfrac{N}{W}$ stores
    \end{itemize}

  \end{answer}
\end{frame}

\begin{frame}[fragile]
  \frametitle{Exercise 5: Models and measurements}
  \begin{enumerate}[itemsep=8pt]
  \item Split into small groups
  \item Make sure one person per group has access to Hamilton
  \item Download the STREAM TRIAD benchmark
  \item Compile with \verb#likwid# annotations
  \item Measure loads and stores
  \item Ask questions!
  \end{enumerate}
\end{frame}

\begin{frame}
  \frametitle{Larger code}
  \begin{challenge}{Problem}
    What if you don't know which part of the code takes all the time?
  \end{challenge}
  \begin{answer}{Answer}
    Use \emph{profiling} to determine hotspots (regions of code where
    all the time is spent).

    $\Rightarrow$ allows us to focus in on important parts.
  \end{answer}
\end{frame}

\begin{frame}
  \frametitle{Profiling: types}
  \begin{itemize}
  \item Goal is to gather information about what a code is doing
    \begin{itemize}
    \item \emph{Sampling}
    \item or \emph{code instrumentation}
    \end{itemize}
  \end{itemize}
  \begin{columns}
    \begin{column}{0.45\textwidth}
      \begin{exampleblock}{Sampling}
        \begin{itemize}
        \item Works with unmodified executables
        \item Only a statistical model of code execution
        \item[$\Rightarrow$] not very detailed for volatile metrics
        \item[$\Rightarrow$] needs long-running application
        \end{itemize}
      \end{exampleblock}
    \end{column}
    \begin{column}{0.45\textwidth}
      \begin{exampleblock}{Instrumentation}
        \begin{itemize}
        \item Requires source code annotations to capture
          ``interesting'' information
        \item Much more details and focused
        \item[$\Rightarrow$] Preprocessing of source required
        \item[$\Rightarrow$] Can have large \emph{overheads} for small functions.
        \end{itemize}
      \end{exampleblock}
    \end{column}
  \end{columns}
\end{frame}

\begin{frame}
  \frametitle{Sampling}
  \begin{itemize}
  \item Running program is periodically interrupted to take a measurement.
  \item Records which function we are in.
  \end{itemize}
  \begin{center}
    \includegraphics[height=0.6\textheight]{figures/samplingprofile}
  \end{center}
\end{frame}
\begin{frame}
  \frametitle{Tracing}
  \begin{itemize}
  \item Measurement code is inserted to capture all the events we care
    about
  \end{itemize}
  \begin{center}
    \includegraphics[height=0.7\textheight]{figures/tracingprofile}
  \end{center}
\end{frame}


\begin{frame}
  \frametitle{Sampling profiles with gprof}
  \begin{exampleblock}{Workflow}
    \begin{enumerate}
    \item Compile \emph{and link} code with symbols (add \texttt{-g}) and profile
      information (\texttt{-p}).
    \item Run code $\Rightarrow$ produces file \texttt{gmon.out}
    \item Postprocess data with \texttt{gprof}
    \item Look at results
    \end{enumerate}
  \end{exampleblock}
\end{frame}

\begin{frame}[fragile]
  \frametitle{gprof ``flat profile''}
\begin{minted}[fontsize=\scriptsize]{sh}
Flat profile:

Each sample counts as 0.01 seconds.
  %   cumulative   self              self     total
 time   seconds   seconds    calls   s/call   s/call  name
 76.14      5.71     5.71      102     0.06     0.06  ForceLJ::compute(Atom&, Neighbor&, Comm&, int)
 17.07      6.99     1.28        6     0.21     0.22  Neighbor::build(Atom&)
  2.80      7.20     0.21        3     0.07     0.07  void ForceLJ::compute_halfneigh<1, 1>(Atom&, Neighbor&, int)
  1.47      7.31     0.11        1     0.11     7.05  Integrate::run(Atom&, Force*, Neighbor&, Comm&, Thermo&, Timer&)
  0.93      7.38     0.07                             __intel_avx_rep_memcpy
  0.40      7.41     0.03       11     0.00     0.00  Neighbor::binatoms(Atom&, int)
  0.40      7.44     0.03        6     0.01     0.01  Comm::borders(Atom&)
  0.40      7.47     0.03        1     0.03     0.04  create_atoms(Atom&, int, int, int, double)
  0.13      7.48     0.01   285585     0.00     0.00  Atom::unpack_border(int, double*)
\end{minted}
\end{frame}
\begin{frame}
  \frametitle{gprof ``flat profile''}
  \begin{itemize}
  \item Code is instrumented (instructions inserted so we know which
    function we're in), triggering of measurement is sampling based
    (not every call).
  \item GProf provides profile with some tracing information
  \item Gives both \emph{inclusive} and \emph{exclusive} timings.
  \end{itemize}
  \begin{columns}
    \begin{column}{0.45\textwidth}
      \begin{itemize}
      \item Blue box shows ``inclusive'' time for \texttt{main}
      \item \texttt{foo} and \texttt{bar} calls (orange) excluded for
        ``exclusive'' time.
      \item[$\Rightarrow$] exclusive time measures execution in
        function that is not attributable to some other function.
      \end{itemize}
    \end{column}
    \begin{column}{0.45\textwidth}
      \begin{center}
        \includegraphics[width=\textwidth]{figures/inclusiveexclusive}
      \end{center}
    \end{column}
  \end{columns}
\end{frame}

\begin{frame}
  \frametitle{Continued workflow}
  \begin{itemize}
  \item After we have identified the hotspot that takes all the time,
    we'd like to determine if it is optimised
  \item[$\Rightarrow$] need more detailed insights.
  \end{itemize}
  \begin{enumerate}
  \item Find relevant bit of code
  \item Determine algorithm
  \item Add instrumentation markers (see exercise)
  \item Profile with more detail/use performance models.
  \item[$\Rightarrow$] guidance for appropriate optimisation.
  \end{enumerate}
\end{frame}
\begin{frame}
  \frametitle{Exercise: finding the hotspot}
  \begin{itemize}
  \item So far, we've looked at very simple code. Now, your task will
    be to find the hotspot and do some exploration in a larger one.
  \item[$\Rightarrow$] Exercise 6 from the usual place.
  \end{itemize}
\end{frame}
\end{document}
